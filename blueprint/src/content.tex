\paragraph{Scope.}
This document is intended to be a readable companion to the Lean 4 development in the repository
root.
It presents the same headline results as the public Lean API
\path{FiniteDependence/API/Definitions.lean} and
\path{FiniteDependence/API/MainTheorems.lean}.
In particular, it proves:
\begin{itemize}
  \item equivalence of the cut and noncontiguous notions of finite dependence on $\Z$;
  \item proper colorings: existence of stationary $1$-dependent $4$-colorings, but no stationary
  $0$-dependent proper $4$-coloring;
  \item proper colorings: existence of stationary $2$-dependent $3$-colorings, but no stationary
  $1$-dependent proper $3$-coloring;
  \item weak $2$-colorings (forbidden \w{000}, \w{111}): existence at dependence range $3$,
  impossibility at range $2$;
  \item greedy proper $3$-colorings: existence at dependence range $6$, impossibility at range $5$;
  \item MIS: existence at dependence range $6$, impossibility at range $5$.
\end{itemize}
The existence results are short reductions from Holroyd--Liggett \cite{HolroydLiggettFDC}.
For early examples separating finite dependence from block-factor structure, see
Burton--Goulet--Meester \cite{BurtonGouletMeester1993}.
The MIS $k=5$ lower bound is presented at the level of finite cylinder probabilities, with exact
verification scripts in \texttt{proof/scripts/}.

\paragraph{Main statements (API order).}
The public Lean API exposes exactly the following headline theorems, proved here as:
\begin{itemize}
  \item cut/noncontiguous dependence equivalence:
  Proposition~\ref{prop:kdep-equivalence};
  \item $4$-colorings ($k=1$ exists, $k=0$ impossible):
  Corollary~\ref{cor:4color-threshold};
  \item $3$-colorings ($k=2$ exists, $k=1$ impossible):
  Corollary~\ref{cor:3color-threshold};
  \item weak $2$-colorings ($k=3$ exists, $k=2$ impossible):
  Corollary~\ref{cor:weak2-threshold};
  \item greedy proper $3$-colorings ($k=6$ exists, $k=5$ impossible):
  Corollary~\ref{cor:greedy3-threshold};
  \item MIS ($k=6$ exists, $k=5$ impossible):
  Corollary~\ref{cor:mis-threshold}.
\end{itemize}

\begin{remark}[API wrapper predicates used in theorem statements]
  \lean{FiniteDependence.IsStationaryKDependentColoring,
    FiniteDependence.ExistsStationaryKDependentColoring,
    FiniteDependence.IsStationaryKDependentWeakTwoColoring,
    FiniteDependence.ExistsStationaryKDependentWeakTwoColoring,
    FiniteDependence.IsStationaryKDependentGreedyThreeColoring,
    FiniteDependence.ExistsStationaryKDependentGreedyThreeColoring,
    FiniteDependence.IsStationaryKDependentMIS,
    FiniteDependence.ExistsStationaryKDependentMIS}
The API theorems are phrased with these bundled predicates:
stationarity + cut-form finite dependence + almost-sure support on the target constraint
(proper coloring, weak 2-coloring, greedy proper 3-coloring, or MIS).
\end{remark}

\section{Problem statement and basic definitions}

We view $\Z$ as the vertex set of the integer line graph, with edges between $i$ and $i+1$.

\begin{definition}[MIS on $\Z$]
\label{def:mis}
  \lean{FiniteDependence.IsMIS, FiniteDependence.No11, FiniteDependence.No000}
  \leanok
For a configuration $x\in\{0,1\}^{\Z}$ we say:
\begin{itemize}
  \item $x$ is an \emph{independent set} if there are no adjacent $1$'s, i.e. $x_i=x_{i+1}=1$ never
  occurs;
  \item $x$ is a \emph{maximal independent set} (MIS) if it is an independent set and moreover one
  cannot add any additional vertex: for every $i$ with $x_i=0$, at least one of $x_{i-1},x_{i+1}$ is
  $1$.
\end{itemize}
\end{definition}

It is convenient to express MIS as a shift of finite type.

\begin{proposition}[MIS as forbidden patterns]
\label{prop:mis-sft}
  \lean{FiniteDependence.IsMIS, FiniteDependence.No11, FiniteDependence.No000}
  \leanok
A configuration $x\in\{0,1\}^{\Z}$ is an MIS if and only if it forbids the two finite patterns
\[
\w{11},\qquad \w{000},
\]
meaning that neither word appears as a contiguous subword of $x$.
\end{proposition}

\begin{proof}
If $\w{11}$ occurs then $x$ is not independent.  If $\w{000}$ occurs at sites $i-1,i,i+1$, then the
middle site $i$ has both neighbors $0$, so adding $i$ preserves independence, contradicting
maximality.

Conversely, if $\w{11}$ never occurs then $x$ is independent.  If $x_i=0$ then the triple
$(x_{i-1},x_i,x_{i+1})$ is not $\w{000}$, hence at least one neighbor is $1$; thus no vertex can be
added and $x$ is maximal.
\end{proof}

\subsection{Stationary processes and cylinders}

Let $(X_i)_{i\in\Z}$ be a random process taking values in $\{0,1\}$.
For a finite binary word $w=w_0\cdots w_{L-1}\in\{0,1\}^L$, define the origin cylinder probability
\[
P(w) := \PP(X_0\cdots X_{L-1}=w).
\]
When the law is stationary, this equals the probability of seeing $w$ starting at \emph{any} site.

\begin{definition}[Stationarity]
  \lean{FiniteDependence.IsStationary}
  \leanok
Let $\mu$ be a probability measure on $\{0,1\}^{\Z}$ and let $T$ be the left shift
$(Tx)_i := x_{i+1}$.  We call $\mu$ \emph{stationary} if $\mu\circ T^{-1} = \mu$.
Equivalently, all cylinder probabilities $\PP(X_a\cdots X_{a+L-1}=w)$ are independent of $a$.
\end{definition}

Stationarity implies the basic ``prefix = suffix'' identities:
for every word $u\in\{0,1\}^{L-1}$,
\begin{equation}
\label{eq:stationarity-prefix-suffix}
\sum_{b\in\{0,1\}} P(ub) = \sum_{b\in\{0,1\}} P(bu).
\end{equation}
(Both sides equal $\PP(X_0\cdots X_{L-2}=u)$, computed either as a prefix marginal or a suffix
marginal.)

\subsection{\texorpdfstring{Two notions of $k$-dependence}{Two notions of k-dependence}}

We now define two standard notions of finite dependence for $\Z$-indexed processes.
Let $\mathcal{F}_A := \sigma(X_i : i\in A)$ be the coordinate $\sigma$-field generated by indices
in $A\subseteq\Z$.

\begin{definition}[Distance between index sets]
  \lean{FiniteDependence.IndexSeparated}
For $A,B\subseteq\Z$ define
\[
\mathrm{dist}(A,B) := \inf\{|a-b| : a\in A, b\in B\}\in\{0,1,2,\dots\}\cup\{\infty\}.
\]
\end{definition}

\begin{definition}[$k$-dependence: noncontiguous (index-set) form]
\label{def:kdep-noncontig}
  \lean{FiniteDependence.IsKDependentNoncontigConfig}
The process $(X_i)_{i\in\Z}$ is \emph{$k$-dependent} in the noncontiguous form if for all
$A,B\subseteq\Z$ with $\mathrm{dist}(A,B) > k$, the $\sigma$-fields $\mathcal{F}_A$ and
$\mathcal{F}_B$ are independent.
\end{definition}

\begin{definition}[$k$-dependence: cut form]
\label{def:kdep-cut}
  \lean{FiniteDependence.IsKDependentCut}
The process is \emph{$k$-dependent} in the cut form if for every $i\in\Z$ the ``past'' and
``future'' $\sigma$-fields
\[
\mathcal{F}_{(-\infty,i]}\quad\text{and}\quad \mathcal{F}_{[i+k+1,\infty)}
\]
are independent.
\end{definition}

Equivalent indexing convention used in Lean:
past on indices $< i$ and future on indices $\ge i+k$.
This is the same notion after the shift $i\mapsto i+1$.

\begin{proposition}[Cut and noncontiguous definitions agree on $\Z$]
\label{prop:kdep-equivalence}
  \lean{FiniteDependence.isKDependentCut_iff_isKDependentNoncontigConfig}
  \leanok
  \uses{def:kdep-noncontig,def:kdep-cut}
Definitions~\ref{def:kdep-noncontig} and~\ref{def:kdep-cut} are equivalent.
In Lean this is stated for all `Fin q`-valued processes (hence applies to all finite color alphabets
used below).
\end{proposition}

\begin{proof}
  \leanok
Noncontiguous $\Rightarrow$ cut: apply Definition~\ref{def:kdep-noncontig} with
$A=(-\infty,i]$ and $B=[i+k+1,\infty)$, whose distance is $k+1$.

Cut $\Rightarrow$ noncontiguous is more subtle: one first proves independence for finite index sets
by repeatedly applying cut-independence across a suitable sequence of cuts, then extends to
arbitrary sets by directed-$\sigma$-field (equivalently monotone-class) arguments.
This is exactly what is formalized in Lean in
\path{FiniteDependence/Coloring/DependenceEquivalence.lean}.
\end{proof}

\subsection{\texorpdfstring{A cylinder identity from $k$-dependence}{A cylinder identity from k-dependence}}

The lower-bound proofs use $k$-dependence only through a finite-word identity.

\begin{lemma}[Prefix/suffix independence identity]
\label{lem:split-identity}
  \lean{FiniteDependence.MIS.Model.KDependent.prob_inter_cyl_eq_mul}
Assume $(X_i)$ is $k$-dependent.
Let $x\in\{0,1\}^m$ and $y\in\{0,1\}^n$ and set $L=m+k+n$.
Then
\begin{equation}
\label{eq:split-identity}
\sum_{g\in\{0,1\}^k} P(xgy) = P(x)\,P(y).
\end{equation}
\end{lemma}

\begin{proof}
Let $A$ be the event $\{X_0\cdots X_{m-1}=x\}$ and let $B$ be the event
$\{X_{m+k}\cdots X_{m+k+n-1}=y\}$.  The index sets $\{0,\dots,m-1\}$ and
$\{m+k,\dots,m+k+n-1\}$ have distance $k+1>k$, so $A$ and $B$ are independent and
$\PP(A\cap B)=\PP(A)\PP(B)=P(x)P(y)$.

On the other hand, $A\cap B$ is the disjoint union (over $g\in\{0,1\}^k$) of the cylinder events
$\{X_0\cdots X_{L-1}=xgy\}$.  Summing their probabilities gives~\eqref{eq:split-identity}.
\end{proof}

\section{Coloring and MIS existence results}

The core existence input is Theorem~1 of Holroyd--Liggett \cite{HolroydLiggettFDC}.

\begin{definition}[Proper coloring on $\Z$]
\label{def:proper-coloring}
  \lean{FiniteDependence.IsColoring}
A process $(C_i)_{i\in\Z}$ with values in $\{0,\dots,q-1\}$ is \emph{proper} if
$C_i\neq C_{i+1}$ for all $i\in\Z$.
\end{definition}

\begin{theorem}[Holroyd--Liggett]
\label{thm:hl-colorings}
  \lean{FiniteDependence.exists_stationary_oneDependent_fourColoring,
    FiniteDependence.exists_stationary_twoDependent_threeColoring}
  \leanok
  \uses{def:proper-coloring}
There exist stationary proper colorings on $\Z$ with:
\begin{itemize}
  \item a $1$-dependent $4$-coloring;
  \item a $2$-dependent $3$-coloring.
\end{itemize}
\end{theorem}

\begin{corollary}[API existence statements for proper colorings]
\label{cor:api-coloring-existence}
  \lean{FiniteDependence.exists_stationary_oneDependent_fourColoring,
    FiniteDependence.exists_stationary_twoDependent_threeColoring}
  \leanok
  \uses{thm:hl-colorings}
There exists a stationary proper $4$-coloring with dependence range $1$, and there exists a
stationary proper $3$-coloring with dependence range $2$.
\end{corollary}

\begin{proof}
  \leanok
This is exactly Theorem~\ref{thm:hl-colorings}.
\end{proof}

\begin{remark}
Beyond Theorem~\ref{thm:hl-colorings}, explicit finitely dependent colorings on $\Z$ include the
symmetric $1$-dependent $q$-colorings for all $q\ge 4$ constructed by Holroyd--Liggett
\cite{HolroydLiggettSymmetric1Dep}. We do not use these additional families in the proofs below.
\end{remark}

\subsection{\texorpdfstring{No stationary $0$-dependent proper $4$-coloring}{No stationary 0-dependent proper 4-coloring}}

\begin{theorem}
\label{thm:no-k0-4coloring}
  \lean{FiniteDependence.not_exists_stationary_zeroDependent_fourColoring}
  \leanok
There is no stationary $0$-dependent proper $4$-coloring process on $\Z$.
\end{theorem}

\begin{proof}
  \leanok
Suppose $(C_i)_{i\in\Z}$ is stationary, proper, and $0$-dependent, with values in $\{1,2,3,4\}$.
For each color $a$, let
\[
A_a:=\{C_0=a\},\qquad B_a:=\{C_1=a\}.
\]
Properness implies $A_a\cap B_a=\varnothing$, hence $\PP(A_a\cap B_a)=0$.

By $0$-dependence, $\sigma(C_0)$ and $\sigma(C_1)$ are independent, so
\[
0=\PP(A_a\cap B_a)=\PP(A_a)\PP(B_a).
\]
Stationarity gives $\PP(A_a)=\PP(B_a)$, therefore $\PP(A_a)^2=0$ and $\PP(A_a)=0$ for each $a$.
Summing over $a\in\{1,2,3,4\}$,
\[
1=\sum_{a=1}^4 \PP(A_a)=0,
\]
contradiction.
\end{proof}

\begin{corollary}[Threshold for proper $4$-colorings]
\label{cor:4color-threshold}
  \lean{FiniteDependence.exists_stationary_oneDependent_fourColoring,
    FiniteDependence.not_exists_stationary_zeroDependent_fourColoring}
  \leanok
  \uses{cor:api-coloring-existence,thm:no-k0-4coloring}
For proper $4$-colorings on $\Z$, stationary finite dependence is possible at range $1$ and
impossible at range $0$.
\end{corollary}

\begin{proof}
  \leanok
Combine Corollary~\ref{cor:api-coloring-existence} (the $4$-coloring existence part) with
Theorem~\ref{thm:no-k0-4coloring}.
\end{proof}

\subsection{A greedy local map from a 3-coloring to an MIS}

Given a proper coloring $c\in\{0,1,2\}^{\Z}$ we define a $0/1$ configuration
$x=\mathrm{GreedyMIS}(c)\in\{0,1\}^{\Z}$ by processing colors in order $0,1,2$:
\begin{itemize}
  \item accept every vertex of color $0$;
  \item accept a vertex of color $1$ if neither neighbor was accepted (equivalently, if neither
  neighbor has color $0$);
  \item accept a vertex of color $2$ if neither neighbor was accepted (equivalently, if neither
  neighbor has color $0$ and neither neighbor is a color-$1$ vertex that gets accepted).
\end{itemize}
Concretely, acceptance at site $i$ depends only on the colors in the window
$\{i-2,i-1,i,i+1,i+2\}$ (radius $2$).
This is exactly the local rule implemented in the Lean file
\path{FiniteDependence/Coloring/MIS.lean}.

\begin{lemma}[Greedy map produces an MIS]
\label{lem:greedy-is-mis}
  \lean{FiniteDependence.Coloring.Model.greedyMIS_isMIS_of_isColoring}
  \leanok
If $c$ is a proper coloring of $\Z$, then $x=\mathrm{GreedyMIS}(c)$ is a maximal independent set.
Equivalently, $x$ forbids $\w{11}$ and $\w{000}$.
\end{lemma}

\begin{proof}[Proof sketch]
By construction, whenever a vertex is accepted, its two neighbors are rejected at the moment they
are processed, so adjacent accepted vertices never occur; hence $\w{11}$ is forbidden.

For maximality, if $x_i=0$ then $i$ was rejected because at least one neighbor was accepted earlier
in the greedy order (a neighbor of color $0$ is always accepted; a neighbor of color $1$ is accepted
precisely when it has no adjacent $0$; etc.).  Thus every rejected vertex has an accepted neighbor,
so $\w{000}$ is forbidden.

A fully formal proof appears in \path{FiniteDependence/Coloring/MIS.lean}.
\end{proof}

\subsection{Dependence range}

Local maps preserve stationarity, and they enlarge the dependence range by at most twice the local
radius.

\begin{lemma}[Local maps preserve finite dependence]
\label{lem:local-factor}
  \lean{FiniteDependence.Coloring.indep_map_of_indep_comap}
Let $(Y_i)_{i\in\Z}$ be a $k$-dependent process and suppose $X_i$ is a deterministic function of
$(Y_{i-r},\dots,Y_{i+r})$ for some radius $r$.
Then $(X_i)$ is $(k+2r)$-dependent.
In Lean this is proved in concrete instances using the generic independence-pushforward helper
\texttt{indep\_map\_of\_indep\_comap}.
\end{lemma}

\begin{proof}
If $A,B\subseteq\Z$ have $\mathrm{dist}(A,B)>k+2r$, then the $r$-neighborhoods
$A^{+r}:=\{a-r,\dots,a+r : a\in A\}$ and $B^{+r}$ satisfy $\mathrm{dist}(A^{+r},B^{+r})>k$.
The $\sigma$-field generated by $(X_i)_{i\in A}$ is contained in that generated by $(Y_j)_{j\in
A^{+r}}$, and similarly for $B$.  Independence of the latter (by $k$-dependence of $Y$) implies
independence of the former.
\end{proof}

Combining Theorem~\ref{thm:hl-colorings} (its $2$-dependent $3$-coloring part) with the radius-$2$
greedy map gives:

\begin{theorem}[Existence of a stationary 6-dependent MIS on $\Z$]
\label{thm:upper-bound}
  \lean{FiniteDependence.exists_stationary_sixDependent_MIS}
  \leanok
  \uses{thm:hl-colorings,lem:greedy-is-mis,lem:local-factor}
There exists a stationary $6$-dependent probability measure on MIS configurations in $\{0,1\}^{\Z}$.
\end{theorem}

\begin{proof}
  \leanok
Let $(C_i)$ be the stationary $2$-dependent proper $3$-coloring from
Theorem~\ref{thm:hl-colorings}.  Define $X=\mathrm{GreedyMIS}(C)$ by the greedy radius-$2$ map.
By Lemma~\ref{lem:greedy-is-mis}, $X$ is almost surely an MIS.
By Lemma~\ref{lem:local-factor} with $(k,r)=(2,2)$, the process $X$ is $6$-dependent, and it is
stationary because it is a shift-commuting factor of a stationary process.
\end{proof}


\section{Lower bound: no stationary 5-dependent MIS}

For the threshold result, it is enough to rule out dependence range $5$.
Indeed, finite dependence is monotone in the parameter:
if a process is $k$-dependent, then it is $k'$-dependent for every $k'\ge k$.
Hence non-existence at $k=5$ automatically implies non-existence for all $k\le 5$.

\begin{remark}[Optional stronger intermediate lower bounds]
  \lean{FiniteDependence.MIS.Model.no_stationary_kDependent_le2,
    FiniteDependence.MIS.no_stationary_threeDependent_MIS,
    FiniteDependence.MIS.g_of_four_equations,
    FiniteDependence.MIS.no_k4_kernel}
The Lean development also formalizes direct contradictions for $k\le 2$ and $k=3$, and the
algebraic $k=4$ kernel used in an optional intermediate argument.
These are not needed for the final threshold statement.
\end{remark}

\subsection{No stationary 5-dependent MIS (computer-assisted, exact)}

\begin{theorem}
\label{thm:no-k5}
  \lean{FiniteDependence.not_exists_stationary_fiveDependent_MIS}
  \leanok
There is no stationary 5-dependent MIS process on $\Z$.
The formal Lean entrypoint for this lower bound is
\path{FiniteDependence/MIS/LowerBounds/K5/Bridge.lean}.
\end{theorem}

\begin{proof}[Proof (with exact computation)]
  \leanok
Assume $k=5$.
Let
\[
p := \PP(X_0=1),\qquad t := \Pw{1010101}.
\]
The proof is a finite, exact contradiction about cylinder probabilities.
Conceptually, stationarity and 5-dependence impose a large but \emph{linear} constraint system on
finite-word probabilities; inconsistency of that system rules out the process.

\paragraph{Complexity bookkeeping.}
The MIS shift forbidding $\w{11}$ and $\w{000}$ has Fibonacci-many admissible words:
the number of allowed length-$L$ words is $2,3,4,5,7,9,12,\dots$ (in particular, $12$ at $L=7$ and
$1897$ at $L=25$).  The argument below uses only sparse split-row cancellations and exact
polynomial algebra over $\mathbb{Q}$.

\paragraph{Step 1: forced length-$7$ family.}
Stationarity, the MIS constraint, and independence of $X_0$ and $X_6$ (distance $6>5$) determine an
affine 2-parameter family of length-$7$ cylinder probabilities, parameterized by $(p,t)$.
Concretely, the allowed length-$7$ words are the following $12$ words, with:
\[
\begin{aligned}
\Pw{0010010}&=p^2-t,\\
\Pw{0010100}&=-2p^2-7p+3t+3,\\
\Pw{0010101}&=p^2+5p-2t-2,\\
\Pw{0100100}&=p^2-t,\\
\Pw{0100101}&=-p^2-2p+t+1,\\
\Pw{0101001}&=-p^2-2p+t+1,\\
\Pw{0101010}&=p^2+5p-t-2,\\
\Pw{1001001}&=p^2-t,\\
\Pw{1001010}&=-p^2-2p+t+1,\\
\Pw{1010010}&=-p^2-2p+t+1,\\
\Pw{1010100}&=p^2+5p-2t-2,\\
\Pw{1010101}&=t.
\end{aligned}
\]
(All other length-$7$ cylinders are $0$ because they are not MIS-admissible.)
This small linear system is solved exactly and printed by \texttt{proof/scripts/prove\_k5\_impossible.py}.

\paragraph{Step 2: two explicit compatibility certificates.}
Starting from the forced length-$7$ family, we first extend exactly to length $15$.
Now build two \emph{fixed} overdetermined split systems whose RHS is computed from these
length-$15$ marginals:
\begin{itemize}
  \item length $16$: $151$ unknown cylinders, $727$ equations;
  \item length $20$: $465$ unknown cylinders, $3002$ equations.
\end{itemize}
For a split row, write
\[
E_{m,x;n,y}:\quad
\sum_{g\in\{0,1\}^5\;:\;xgy\text{ admissible}}\Pw{xgy}=\Pw{x}\Pw{y}.
\]
Every $E_{m,x;n,y}$ is one row of the corresponding fixed system.

\emph{Certificate for $f(p,t)=0$ (length $16$).}
In the length-$16$ system, consider
\[
\begin{aligned}
\mathcal C_f:=\;&
E_{1,\w{1};10,\w{0010100101}}
-E_{3,\w{100};8,\w{10100101}}
+E_{5,\w{10100};6,\w{001001}}\\
&-E_{5,\w{10100};6,\w{100101}}
+E_{6,\w{100100};5,\w{00101}}
+E_{8,\w{10100101};3,\w{101}}
-E_{10,\w{1010010101};1,\w{1}}.
\end{aligned}
\]
The left-hand side of $\mathcal C_f$ is an exact telescoping cancellation:
there are $12$ split-word terms before cancellation and $0$ after cancellation.
Hence the same signed combination of right-hand sides must vanish, giving
\[
f(p,t)=0.
\]

\emph{Certificate for $r(p,t)=0$ (length $20$).}
In the length-$20$ system, consider
\[
\begin{aligned}
\mathcal C_r:=\;&
-E_{5,\w{00100};10,\w{1010010100}}
+E_{8,\w{00100100};7,\w{0010100}}\\
&+E_{10,\w{0010010100};5,\w{10100}}
-E_{13,\w{0010010100100};2,\w{00}}.
\end{aligned}
\]
Again the left-hand side cancels exactly (here: $6$ split-word terms to $0$), so the signed RHS
combination vanishes:
\[
r(p,t)=0.
\]
Thus Step~2 uses only sparse row-combination checks (certificates), not solving the
$727\times 151$ or $3002\times 465$ fixed systems.

Therefore any stationary 5-dependent MIS must satisfy
\[
\begin{aligned}
f(p,t) &=
-3p^4-20p^3+6p^2t-48p^2+30pt+45p-3t^2-12t-9,\\
r(p,t) &=
-16p^4+12p^2t+84p^2-60pt-60p+9t^2+21t+11.
\end{aligned}
\]
Hence
\[
g(p,t):=3f(p,t)+r(p,t)
=-25p^4-60p^3+30p^2t-60p^2+30pt+75p-15t-16
\]
must also vanish.

\paragraph{Step 3: eliminate $t$.}
The polynomial $g$ is linear in $t$, so one can solve explicitly:
\[
t=\frac{25 p^4+60 p^3+60 p^2-75 p+16}{15(2p^2+2p-1)}.
\]
Substituting this into $f$ and clearing denominators yields
\[
A(p)B(p)=0,
\]
where
\[
\begin{aligned}
A(p)&=5p^4-355p^3+460p^2-200p+29,\\ B(p)&=5p^4-5p^3-5p^2+5p-1.
\end{aligned}
\]
Hence any putative process must satisfy either $A(p)=0$ or $B(p)=0$.

\paragraph{Step 4: a sparse length-$19/25$ certificate gives $q(p,t)=0$.}
Set
\[
\begin{aligned}
Y_{19}&:=\Pw{0010101010010010100},\\
S_{25}&:=\Pw{0101010010101010010010100},\\
A_{25}&:=\Pw{0010010010101010010010100},\\
T_{25}&:=\Pw{0010010100101010010010100}.
\end{aligned}
\]
From two length-$19$ split rows (supports of sizes $1$ and $2$):
\[
\Pw{0010101001010010100}=\Pw{00}\,\Pw{001010010100},
\]
\[
\Pw{0010101001010010100}+Y_{19}
=\Pw{0010101}\,\Pw{0010100},
\]
we get
\[
Y_{19}=\Pw{0010101}\,\Pw{0010100}-\Pw{00}\,\Pw{001010010100}.
\]

From two length-$25$ split rows (again supports of sizes $1$ and $2$):
\[
\Pw{0101010010101001010010100}
=\Pw{01010100}\,\Pw{001010010100},
\]
\[
\Pw{0101010010101001010010100}+S_{25}
=\Pw{0101010010101}\,\Pw{0010100},
\]
we get
\[
S_{25}
=\Pw{0101010010101}\,\Pw{0010100}
-\Pw{01010100}\,\Pw{001010010100}.
\]

Now use the three length-$25$ rows (supports of sizes $2,2,1$):
\[
\begin{aligned}
\text{(A)}\quad A_{25}+S_{25}&=\Pw{0}\,Y_{19}=(1-p)\,Y_{19},\\
\text{(B)}\quad A_{25}+T_{25}&=\Pw{00100}\,\Pw{101010010010100},\\
\text{(W)}\quad T_{25}&=\Pw{0010010100}^{\,2}.
\end{aligned}
\]
Eliminating $A_{25},T_{25}$ gives
\[
(1-p)Y_{19}-S_{25}-\Pw{00100}\,\Pw{101010010010100}+\Pw{0010010100}^{\,2}=0.
\]
Substituting the formulas above for $Y_{19},S_{25}$ and the forced length-$15$ marginals
yields a third compatibility polynomial
\[
\begin{aligned}
q(p,t):={}&4p^6+64p^5-12p^4t+528p^4-240p^3t+80p^3+12p^2t^2\\
&-576p^2t-552p^2+180pt^2+540pt+260p-4t^3-72t^2-108t-35=0.
\end{aligned}
\]

\paragraph{Step 5: eliminate $t$ and compare univariate constraints.}
From Step~3:
\[
A(p)B(p)=0.
\]
Substituting
\[
t=\frac{25 p^4+60 p^3+60 p^2-75 p+16}{15(2p^2+2p-1)}
\]
into $q$ gives
\[
q\bigl(p,t(p)\bigr)=
\frac{P_{12}(p)}{3375(2p^2+2p-1)^3},
\]
where
\[
\begin{aligned}
P_{12}(p):={}&500p^{12}-306000p^{11}+976500p^{10}+6273000p^9-11736300p^8-1616400p^7\\
&+16300800p^6-13617000p^5+4003260p^4+377640p^3-541710p^2+129690p-10579.
\end{aligned}
\]
Hence any putative process must satisfy $P_{12}(p)=0$ as well.
Exact Euclidean-algorithm checks give
\[
\gcd(A,P_{12})=\gcd(B,P_{12})=1,
\qquad
\gcd\!\bigl(A,2p^2+2p-1\bigr)=\gcd\!\bigl(B,2p^2+2p-1\bigr)=1.
\]
So $A(p)B(p)=0$ and $P_{12}(p)=0$ are incompatible, contradiction.

Therefore no stationary 5-dependent MIS can exist.
\end{proof}

\begin{corollary}[Threshold for MIS]
\label{cor:mis-threshold}
  \lean{FiniteDependence.exists_stationary_sixDependent_MIS,
    FiniteDependence.not_exists_stationary_fiveDependent_MIS}
  \leanok
  \uses{thm:upper-bound,thm:no-k5}
For MIS on $\Z$, stationary finite dependence is possible at range $6$ and impossible at range $5$.
\end{corollary}

\begin{proof}
  \leanok
Combine Theorem~\ref{thm:upper-bound} with Theorem~\ref{thm:no-k5}.
\end{proof}

\section{\texorpdfstring{Weak 2-coloring (forbidden \w{000},\w{111}): threshold $k=3$}{Weak 2-coloring (forbidden 000,111): threshold k=3}}

\begin{definition}[Weak 2-coloring on $\Z$]
\label{def:weak2}
  \lean{FiniteDependence.IsWeakTwoColoring, FiniteDependence.No111}
A binary process $(X_i)_{i\in\Z}$ is a \emph{weak 2-coloring process} if it forbids
\[
\w{000},\qquad \w{111}.
\]
\end{definition}

\begin{theorem}[Existence at dependence range $3$]
\label{thm:weak2-exists-k3}
  \lean{FiniteDependence.exists_stationary_threeDependent_weakTwoColoring}
  \leanok
  \uses{thm:hl-colorings,def:weak2}
There exists a stationary $3$-dependent weak 2-coloring process on $\Z$.
\end{theorem}

\begin{proof}
  \leanok
Let $(C_i)_{i\in\Z}$ be the stationary $2$-dependent proper $3$-coloring from
Theorem~\ref{thm:hl-colorings}, with values in $\{0,1,2\}$.
Define
\[
Y_i := \mathbf 1\{C_i < C_{i+1}\}\in\{0,1\}.
\]

If $Y_i=Y_{i+1}=Y_{i+2}=1$, then
\[
C_i<C_{i+1}<C_{i+2}<C_{i+3},
\]
impossible in $\{0,1,2\}$.  So $\w{111}$ is forbidden.

If $Y_i=Y_{i+1}=Y_{i+2}=0$, then
\[
C_i>C_{i+1}>C_{i+2}>C_{i+3},
\]
also impossible in $\{0,1,2\}$.  So $\w{000}$ is forbidden.

Thus $Y$ is a weak 2-coloring process.
It is stationary because it is a shift-commuting factor of $C$.

For dependence, note $Y_i$ depends only on $(C_i,C_{i+1})$.
Hence for index sets $A,B\subseteq\Z$, the $\sigma$-field generated by $(Y_i)_{i\in A}$ is
contained in that generated by $(C_j)_{j\in A\cup(A+1)}$, and similarly for $B$.
If $\mathrm{dist}(A,B)>3$, then $\mathrm{dist}(A\cup(A+1),B\cup(B+1))>2$.
By $2$-dependence of $C$, these enlarged coordinate families are independent, so
$(Y_i)$ is $3$-dependent.
\end{proof}

\begin{theorem}[No dependence range $2$]
\label{thm:weak2-no-k2}
  \lean{FiniteDependence.not_exists_stationary_twoDependent_weakTwoColoring}
  \leanok
  \uses{def:weak2,thm:no-k5}
There is no stationary $2$-dependent weak 2-coloring process on $\Z$.
\end{theorem}

\begin{proof}
  \leanok
Suppose $X=(X_i)_{i\in\Z}$ is stationary, $2$-dependent, and forbids $\w{000},\w{111}$.
Define a one-sided radius-$3$ local map $\Phi:\{0,1\}^4\to\{0,1\}$ by
\[
\Phi(u_0u_1u_2u_3) := \mathbf 1\!\left\{
u_0u_1u_2u_3\in
\{\w{0010},\w{0011},\w{1010},\w{1011},\w{1100}\}
\right\}.
\]
Set
\[
M_i := \Phi(X_iX_{i+1}X_{i+2}X_{i+3}).
\]
So $M_i$ depends only on $(X_i,\dots,X_{i+3})$.

We claim $M$ is always an MIS configuration (forbids $\w{11}$ and $\w{000}$) whenever $X$ forbids
$\w{000},\w{111}$.
This is a finite check on admissible source words:
\begin{itemize}
  \item among admissible length-$5$ words of $X$, none yields $M_iM_{i+1}=\w{11}$;
  \item among admissible length-$6$ words of $X$, none yields $M_iM_{i+1}M_{i+2}=\w{000}$.
\end{itemize}
An explicit exhaustive verifier is included as
\path{proof/scripts/prove_weak2_threshold.py}.

Therefore $M$ is a stationary MIS process.
Since $M_i$ uses the one-sided window $[i,i+3]$, a $k$-dependent input gives a $(k+3)$-dependent
output; with $k=2$ this gives that $M$ is $5$-dependent.
This contradicts Theorem~\ref{thm:no-k5} (no stationary $5$-dependent MIS).
So no stationary $2$-dependent weak 2-coloring exists.
\end{proof}

\begin{corollary}[Exact threshold]
\label{cor:weak2-threshold}
  \lean{FiniteDependence.exists_stationary_threeDependent_weakTwoColoring,
    FiniteDependence.not_exists_stationary_twoDependent_weakTwoColoring}
  \leanok
  \uses{thm:weak2-exists-k3,thm:weak2-no-k2}
For weak 2-coloring on $\Z$ (forbidden $\w{000},\w{111}$), finite dependence is possible at
$k=3$ and impossible at $k\le 2$.
\end{corollary}

\begin{proof}
  \leanok
Combine Theorems~\ref{thm:weak2-exists-k3} and~\ref{thm:weak2-no-k2}.
Impossibility at $k\le 2$ follows since $k$-dependence is monotone in $k$
(if a process is $k$-dependent, then it is $k'$-dependent for every $k'\ge k$).
\end{proof}

\section{\texorpdfstring{Greedy proper 3-coloring: threshold $k=6$}{Greedy proper 3-coloring: threshold k=6}}

\begin{definition}[Greedy proper 3-coloring on $\Z$]
\label{def:greedy3}
  \lean{FiniteDependence.IsGreedyThreeColoring}
  \leanok
A process $G=(G_i)_{i\in\Z}$ with values in $\{0,1,2\}$ is a \emph{greedy proper 3-coloring} if:
\begin{itemize}
  \item it is proper: $G_i\neq G_{i+1}$ for all $i$;
  \item if $G_i=1$, then $G_{i-1}=0$ or $G_{i+1}=0$;
  \item if $G_i=2$, then it has a $0$-neighbor and a $1$-neighbor:
  \[
  (G_{i-1}=0\ \text{or}\ G_{i+1}=0)\quad\text{and}\quad
  (G_{i-1}=1\ \text{or}\ G_{i+1}=1).
  \]
\end{itemize}
This is exactly the predicate \texttt{IsGreedyThreeColoring} in
\path{FiniteDependence/API/Definitions.lean}.
\end{definition}

\begin{theorem}[Existence at dependence range $6$]
\label{thm:greedy3-exists-k6}
  \lean{FiniteDependence.exists_stationary_sixDependent_greedyThreeColoring}
  \leanok
  \uses{thm:weak2-exists-k3,def:greedy3}
There exists a stationary $6$-dependent greedy proper $3$-coloring process on $\Z$.
\end{theorem}

\begin{proof}
  \leanok
Let $X=(X_i)_{i\in\Z}$ be the stationary $3$-dependent weak $2$-coloring from
Theorem~\ref{thm:weak2-exists-k3}, so $X$ forbids \w{000} and \w{111}.
Define a local map $\Phi:\{0,1\}^4\to\{0,1,2\}$ by
\[
\Phi(u)=
\begin{cases}
2, & u\in\{\w{0010},\w{1011}\},\\
1, & u\in\{\w{0110},\w{1001},\w{1010}\},\\
0, & \text{otherwise},
\end{cases}
\]
and set
\[
G_i := \Phi(X_iX_{i+1}X_{i+2}X_{i+3}).
\]
So $G_i$ depends only on the one-sided window $[i,i+3]$.

A finite local check on admissible source windows (equivalently, the exhaustive
\texttt{decide}-lemmas formalized in \path{FiniteDependence/Coloring/GreedyThree.lean})
shows: whenever $X$ forbids \w{000},\w{111}, the output $G$ is always a greedy proper $3$-coloring.

Stationarity is preserved by this shift-commuting local factor.
For dependence, one-sided radius $3$ enlarges range by at most $3$, so from $3$-dependence of $X$
we get $6$-dependence of $G$.
\end{proof}

\begin{theorem}[No dependence range $5$]
\label{thm:greedy3-no-k5}
  \lean{FiniteDependence.not_exists_stationary_fiveDependent_greedyThreeColoring}
  \leanok
  \uses{thm:no-k5}
There is no stationary $5$-dependent greedy proper $3$-coloring process on $\Z$.
\end{theorem}

\begin{proof}
  \leanok
Suppose $G=(G_i)_{i\in\Z}$ is stationary, $5$-dependent, and a greedy proper $3$-coloring.
Define a binary process
\[
M_i := \mathbf 1\{G_i=0\}.
\]
This is a sitewise factor, so $M$ is stationary and still $5$-dependent.

Also $M$ is an MIS configuration:
\begin{itemize}
  \item no \w{11}: if $M_i=M_{i+1}=1$, then $G_i=G_{i+1}=0$, contradicting properness of $G$;
  \item no \w{000}: if $M_i=M_{i+1}=M_{i+2}=0$, then $G_{i+1}\in\{1,2\}$.
  The greedy condition at site $i+1$ forces a neighboring $0$, i.e. $G_i=0$ or $G_{i+2}=0$,
  contradiction.
\end{itemize}
Hence $M$ is a stationary $5$-dependent MIS process, contradicting Theorem~\ref{thm:no-k5}.
\end{proof}

\begin{corollary}[Threshold for greedy proper $3$-colorings]
\label{cor:greedy3-threshold}
  \lean{FiniteDependence.exists_stationary_sixDependent_greedyThreeColoring,
    FiniteDependence.not_exists_stationary_fiveDependent_greedyThreeColoring}
  \leanok
  \uses{thm:greedy3-exists-k6,thm:greedy3-no-k5}
For greedy proper $3$-colorings on $\Z$, stationary finite dependence is possible at range $6$ and
impossible at range $5$.
\end{corollary}

\begin{proof}
  \leanok
Combine Theorems~\ref{thm:greedy3-exists-k6} and~\ref{thm:greedy3-no-k5}.
\end{proof}

\section{No stationary 1-dependent proper 3-coloring}

\begin{theorem}
\label{thm:no-k1-3coloring}
  \lean{FiniteDependence.not_exists_stationary_oneDependent_threeColoring}
  \leanok
  \uses{thm:weak2-no-k2}
There is no stationary $1$-dependent proper $3$-coloring process on $\Z$.
\end{theorem}

\begin{proof}
  \leanok
Suppose $(C_i)_{i\in\Z}$ is stationary, proper, and $1$-dependent, with values in $\{0,1,2\}$.
Define the ascent factor
\[
Y_i := \mathbf 1\{C_i<C_{i+1}\}\in\{0,1\}.
\]
Exactly as in the proof of Theorem~\ref{thm:weak2-exists-k3}, $Y$ forbids \w{000} and \w{111},
so $Y$ is a weak $2$-coloring process, and $Y$ is stationary.

For dependence (cut form), fix a cut index $i$.
The past $\sigma$-field generated by $(Y_j)_{j\le i}$ is contained in
$\sigma(C_j:j\le i+1)$, while the future $\sigma$-field generated by $(Y_j)_{j\ge i+3}$ is
contained in $\sigma(C_j:j\ge i+3)$.
By $1$-dependence of $C$ at cut $i+1$, these are independent.
Hence $Y$ is $2$-dependent.

So $Y$ is a stationary $2$-dependent weak $2$-coloring, contradicting
Theorem~\ref{thm:weak2-no-k2}.
\end{proof}

\begin{corollary}[Threshold for proper $3$-colorings]
\label{cor:3color-threshold}
  \lean{FiniteDependence.exists_stationary_twoDependent_threeColoring,
    FiniteDependence.not_exists_stationary_oneDependent_threeColoring}
  \leanok
  \uses{cor:api-coloring-existence,thm:no-k1-3coloring}
For proper $3$-colorings on $\Z$, stationary finite dependence is possible at range $2$ and
impossible at range $1$.
\end{corollary}

\begin{proof}
  \leanok
Combine Corollary~\ref{cor:api-coloring-existence} (the $3$-coloring existence part) with
Theorem~\ref{thm:no-k1-3coloring}.
\end{proof}

\section{How to verify the computations}

The $k=3$ and $k=4$ arguments above are now written as finite pen-and-paper derivations.
Their scripts are optional independent audits.

The $k=5$ argument remains genuinely computer-assisted.
The script \texttt{prove\_k5\_impossible.py} now has two modes:
\begin{itemize}
  \item default mode: a rigorous exact-certificate check of the proof strategy in Steps~1--5:
  it verifies Step~1 identities, builds the forced length-$15$ family exactly, verifies the
  sparse Step~2 certificates by explicit LHS cancellation and exact RHS evaluation, verifies the
  sparse Step~4 length-$19/25$ certificate and polynomial $q(p,t)$, then performs exact
  univariate elimination/GCD checks showing incompatibility of $A(p)B(p)=0$ with $P_{12}(p)=0$;
  \item \texttt{--full-audit}: an additional slower check that also rebuilds the large fixed
  Step~2 systems (sizes $727\times 151$ and $3002\times 465$) before applying the same sparse
  certificates.
\end{itemize}

\paragraph{Independent verification checklist.}
For a minimal independent check of the final theorem:
\begin{itemize}
  \item verify the displayed split identities and algebra in Steps~2--4;
  \item run \texttt{prove\_k5\_impossible.py} (default rigorous mode) and confirm it reproduces:
  the $7$-row length-$16$ certificate $\mathcal C_f$ (LHS cancellation and RHS polynomial $f$),
  the $4$-row length-$20$ certificate $\mathcal C_r$ (LHS cancellation and RHS polynomial $r$),
  the identity $g=3f+r$,
  the factorization $A(p)B(p)=0$ after eliminating $t$,
  the sparse Step~4 polynomial $q(p,t)$,
  the elimination numerator $P_{12}(p)$ from $q\bigl(p,t(p)\bigr)$,
  and the exact checks $\gcd(A,P_{12})=\gcd(B,P_{12})=1$.
  \item optionally run \texttt{prove\_k5\_impossible.py --full-audit} for the extra large-system
  Step~2 rebuild check.
\end{itemize}

Run (from \texttt{proof/}):
\begin{align*}
&\texttt{uv run python scripts/prove\_weak2\_threshold.py}\\
&\texttt{uv run --with sympy python scripts/prove\_k3\_impossible.py}\\
&\texttt{uv run --with sympy python scripts/prove\_k4\_impossible.py}\\
&\texttt{uv run --with sympy python scripts/prove\_k5\_impossible.py}\\
&\texttt{uv run --with sympy python scripts/prove\_k5\_impossible.py --full-audit}
\end{align*}

\bibliographystyle{alpha}
\bibliography{refs}
